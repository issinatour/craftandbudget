% Created 2015-06-05 Fri 06:52
\documentclass[11pt]{article}
\usepackage[utf8]{inputenc}
\usepackage[T1]{fontenc}
\usepackage{fixltx2e}
\usepackage{graphicx}
\usepackage{longtable}
\usepackage{float}
\usepackage{wrapfig}
\usepackage{rotating}
\usepackage[normalem]{ulem}
\usepackage{amsmath}
\usepackage{textcomp}
\usepackage{marvosym}
\usepackage{wasysym}
\usepackage{amssymb}
\usepackage{hyperref}
\tolerance=1000
\date{\today}
\title{craftandbudget}
\hypersetup{
  pdfkeywords={},
  pdfsubject={},
  pdfcreator={Emacs 24.5.1 (Org mode 8.2.10)}}
\begin{document}

\maketitle
\tableofcontents

\section{Implementación de la bases de datos}
\label{sec-1}
Se ha tratado de implementar una base de datos los más completa y
escalable posible. Como se puede apreciar en el modelo, se ha dado
la funcionalidad a la app de que se puedan dar de alta varios
usuarios a la vez en una misma tienda. Para ello se ha creado una
tabla "craftshop$_{\text{users}}$" en la que se relacionan un usuario de la
tabla "user", con un rol de usuario de la tabla "user$_{\text{rol}}$" y con una
tienda. De está manera, además de que puedan haber varios usuarios
en una tienda, se les pueden asignar roles. Siendo el de
administrador el más importante, ya que este será el que tenga los
máximos privilegios. \{\{diferencia entre roles de usuarios\}\}

En cuanto a la tienda al que se conecta, como se puede apreciar en
la tabla "shop" se necesita almacenar el api key de la web a la que
pertence. Esta api key, permite privilegios CRUD en la base de datos
original usando la web api de prestsashop o de la tienda en
cuestión. En nuestra base de datos hay dos tipos de tiendas, las
tiendas virtuales que tienen los ususario online, y la tienda
craftshop. Un usuario puede tener varias tiendas online, pero solo
puede tener una tienda craftshop. Se consigue así que los artesanos
puedan utilizar craft\&shop para sincronizar todas sus tiendas
online, y saber cuantos y que productos tienen online en cada
tienda. Para que cuando los usuarios sepan rápidamente de que tienda
es cada producto, se ha decidido no solo especificar una columna
para el nombre de la tienda virtual, sino que también se ha creado
una columna para almacenar el logo de la tienda. Así, cuando el
usuario este visualizando todos los productos que tiene a la venta,
podrá saber a que tienda pertence cada uno solo con ver la imágen.

Se ha decidido también implementar una página multilenguaje para así
tener más usuarios potenciales. Para ello, se ha creado una tabla
"lang" en donde se especifica cada lenguaje que aceptará craft\&shop
\{\{lista total o parcial\}\}. Se relacionará cada usuario con un o
varios lenguajes. Para saber que lenguajes tiene disponible un
usuario, se usará la web api de prestashop, ya que en su base de
datos se almacena los lenguajes que tiene el usuario. La parte la
página web que será multilenguaje será la referente a los
productos. La idea de implantar está funcionalidad es que muchos
artesanos traducen sus productos al inglés o a otro idioma para
conseguir más clientes. Desde prestashop no es posible averiguar
cuál es su lenguaje principal, así que se optó por descargar todos
sus productos en todos los lenguajes que tenga especificado, y que
sea el propio usuario quien eliga su lenguaje a través de una
pantalla de configuración. Se ha considerado que las únicas partes
multilenguaje de nuestra aplicacón sea los productos, y los datos
estáticos de la página, es decir, los menús y mensajes de
información. Los materiales, y proveedores solo se almacenarán en la
base de datos en un solo lenguaje, ya que nuestra app será un back
office para los artesanos, no un producto final para la vista de los
clientes. Tiene más sentido que la información insertada por el
usuario se almacena tan solo en un lenguaje, el lenguaje principal
del usuario. La tabla que se encargará de almacenar los diferentes
lenguajes de cada producto sera la tabla "product$_{\text{language}}$". Esta
tabla es la relación n-n entre la tabla "product" y  la tabla "lang".

Los materiales se almacenan en la tabla "materiales". Como se ha
dicho anteriormente, los materiales son los materias necesarias para
porder fabricar los productos de los artesanos. Estos materiales
tendrá que ser introducidos por los artesanos, ya que no existen en
las tiendas virtuales. Para que los artesanos puedan tener más
organizado el stock de materiales, se ha optado por relacionar a
cada material con una categoría. Así, los usuarios podrán saber
todos los materiales disponibles de una categoría dada, y podrán
buscar de manera eficiente el stock de materiales. Como nuestro
sistema va a ser multiusuario para una misma tienda, se ha optado
por relacionar cada material no con un usuario, pero con la tienda
craft\&budget que comparten todos los usuarios de una misma
tienda. 

Estos materiales no solo tiene la funcionalidad de especificar
cuantos materiales tiene un usuario, sino que también sirven para
saber cuantos materiales son necesarios para fabricar un producto.
% Emacs 24.5.1 (Org mode 8.2.10)
\end{document}
